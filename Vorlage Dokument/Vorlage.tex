\documentclass[12px, twocolumn, a4paper]{article}
\usepackage{fullpage}
\usepackage[utf8]{inputenc}
\usepackage{ngerman}
\usepackage{blindtext}

\title{Titel der T-1000}
\author{Vorname Nachname}
\begin{document}
\date{\parbox{\linewidth}{\centering
	matrikelNr. 12346654, FIRMA\endgraf
	SNummer@student.dhbw-mannheim.de }}

\maketitle

%Abstract für des Papers
\section*{abstract}
\blindtext %LÖSCHEN BEIM BEARBEITEN

%Einfügen des Inhaltsverzeichnisses
\tableofcontents
\newpage

%Beginn des Dokuments nach dem Inhaltsverzeichnis
\section{Große Überschrift}
\blindtext 

\subsection{Unterüberschrift}
\Blindtext

\section{Überschrift}
\blindtext

\subsection{Unterüberschrift Nr. 1}
\blindtext\bibliography{Quellen}

\subsection{Unterüberschrift Nr. 2}
lkjabHEDLK jadflkh \cite{Abbildung1} aslkdfh \cite{Buchbeispiel} alksdfhj klasdfhklasdh fklas hjdfkash dfhasdklf haslkdfh aslkdfhakl sdhfkas hdfklashdflk hasldkfj hasldkfhadksjhf alksdjh flk shlkh lk hlk hjlkhlkjh lkh lkjh lkjh ljk lk 


\newpage


\bibliographystyle{ieeetr}
\renewcommand{\refname}{Internetquellen}
\bibliography{Internetquellen}
\renewcommand{\refname}{Literaturquellen}
\bibliography{literatur}
\renewcommand{\refname}{Abbildungsverzeichnis}
\bibliography{Abbildungsverzeichnis}
\end{document}






