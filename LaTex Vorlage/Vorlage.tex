\documentclass[12px, a4paper]{article}
\usepackage[
	left=2.5cm,
	right=2.5cm,
	top=3cm,
	bottom=3cm,
]{geometry}
\usepackage[utf8]{inputenc}
\usepackage[ngerman]{babel}
\usepackage{blindtext}
\usepackage{setspace}
\usepackage{rotating}
\usepackage{lmodern}
\usepackage[bottom]{footmisc}
\usepackage[pdfborder={0 0 0}]{hyperref}
\usepackage{graphicx}

\title{TITEL}
\author{Name Nachname}
\begin{document}


%Titelseite
\begin{titlepage}

	\centering
	{\scshape\LARGE Duale Hochschule Baden - Württemberg \par}
	\vspace{1cm}
	{\scshape\Large Projektarbeit \par T1000 1. Studienjahr \par}
	\vspace{1.5cm}
	{\huge\bfseries TITEL\par}
	\vspace{2cm}
	{\Large\itshape NAME NACHNAME\par}
	Studiengang\\
	Matr. Nr., sxxxxxx@student.dhbw-mannheim.de
	\vfill
	Dualer Partner\par
	FIRMA \par
	

	\vfill

% Bottom of the page
	{\large \today\par}
\end{titlepage}

%Sperrvermerk und Themenschwerpunkte in der Praxis
\thispagestyle{empty}
\section*{Sperrvermerk für den Praxisbericht}
\blindtext
\vspace{2cm}
%Erklärung
\section*{Erklärung}
Ich versichere hiermit, dass ich meinen T1000 Praxisbericht selbstständig verfasst habe und keine anderen als die angegebenen Quellen und Hilfsmittel verwendet habe. \\
\\ 
Ich versichere zudem, dass die eingereichte elektronische Fassung mit der gedruckten Fassung übereinstimmt.\footnote{Falls beide Fassungen erforderlich sind} \\
\vspace{2cm}

\parbox{5cm}{\centering \hrule
\strut \centering\footnotesize Ort, Datum} \hfill\parbox{5cm}{\hrule
\strut \centering\footnotesize Name Nachname}
\vspace{2 cm}
\section*{Vorwort}
\blindtext

\newpage
\thispagestyle{empty}
\section*{Bearbeitete Tätigkeitsschwerpunkte }
\begin{enumerate}
	\item INHALT
	\item INHALT
	\item INHALT
	\item INHALT
\end{enumerate}
\begin{spacing}{1.5}
\subsection*{Kurze Beschreibung der 1. Praxisphase}
\Blindtext
 \newpage
%Abstract des Papers
\section*{Abstract}
\blindtext

\section*{Motivation}

\Blindtext

\newpage


\newpage
%-----------------------------------
%Einfügen des Inhaltsverzeichnisses
\tableofcontents
\newpage
%-----------------------------------


\newpage
%Abkürzungsverzeichnis
\section*{Abkürzungsverzeichnis}
\begin{tabular}[h]{l | l}
	\hline
	\textbf{Abkürzung} & \textbf{Bedeutung} \\
	\hline
		\textbf{xyx} 	& Hallo \\
		\textbf{huhs} 	& Test \\
		\textbf{zuy}	& Inhalt \\
		
		
\end{tabular}
\newpage
%Aufgabenstellung mit beginnender Seitenzahl und zweispaltenformat

\setlength{\columnsep}{2.5cm}

\setcounter{page}{1}

\section{Aufgabenstellung}
\subsection{Problemstellung}
\blindtext

\subsection{Geplantes Vorgehen}
\blindtext
\newpage
\section{Grundlagen}
\newpage
%HAUPTTEIL Beginn
\section{Hauptteil}

\subsection{Überschrift}
\blindtext
\subsection{Überschrift}
\blindtext
\subsection{Überschrift}
\blindtext
\subsection{Überschrift}
\blindtext
\subsection{Überschrift}
\blindtext
\subsection{Überschrift}
\blindtext
\subsection{Überschrift}
\blindtext
\subsubsection{Alternative Nr. 1}
\blindtext
\subsubsection{Alternative Nr. 2}
\blindtext
\subsubsection{Alternative Nr. 3}
\blindtext
\subsection{Überschrift}
\blindtext


\newpage
%HAUPTTEIL Ende
\section{Kritische Reflexion und Ausblick}
\Blindtext
\newpage

%-----------------------
%Graphik einfügen Anfang
%\begin{figure}[h]
% \centering
% \includegraphics[width=0.7\textwidth]{Bilder/csm_logo-dhbw_121398bf91}
% \caption{Meine Grafik}
% \label{fig:meine-grafik}
%\end{figure}
%Graphik einfügen Ende
%-----------------------

\newpage

%Internetquellen werden als Fußnote angegeben!
\bibliographystyle{ieeetr}
\section{Quellen}
\renewcommand{\refname}{Literaturverzeichnis}
\bibliography{literatur}
\listoffigures
\end{spacing}
\onecolumn
\newpage
\section{Anänge}


\subsection*{Anhang Nr. 1}
%-----------------------
%Graphik einfügen
 %\centering
 %\includegraphics[angle=90, width=0.65\textwidth]{Bilder/screenshot}
\newpage

\subsection*{Anhang Nr. 2}

\newpage

\end{document}
